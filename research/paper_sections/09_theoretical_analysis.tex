\section{Theoretical Analysis}

\subsection{Quality Scoring Methodology}

Our quality scoring system provides a comprehensive evaluation framework that balances multiple performance criteria:

\begin{equation}
\text{Quality Score} = \alpha \cdot \text{Accuracy} + \beta \cdot \text{Reliability} + \gamma \cdot \text{Efficiency}
\end{equation}

Where the weights are chosen to reflect clinical priorities: $\alpha = 0.5$ (accuracy), $\beta = 0.3$ (reliability), and $\gamma = 0.2$ (efficiency).

\textbf{Accuracy Component}: Based on mean absolute error in Hurst estimation across all contamination scenarios, normalized to a 0-100 scale.

\textbf{Reliability Component}: Success rate across all contamination scenarios, measuring the proportion of valid estimates produced.

\textbf{Efficiency Component}: Processing time efficiency, normalized by the fastest method in each category.

\subsection{Statistical Validation}

The benchmark provides robust statistical validation through:

\textbf{Sample Size}: 945 total tests across 12 estimators and 8 contamination scenarios provide sufficient statistical power for reliable performance comparisons.

\textbf{Contamination Diversity}: Eight realistic contamination types cover the spectrum of data quality issues encountered in clinical practice.

\textbf{Reproducibility}: Standardized testing protocol ensures results are reproducible and comparable across different implementations.

\subsection{Clinical Relevance}

The benchmark design directly addresses clinical requirements:

\textbf{Real-Time Processing}: Sub-100ms processing times for continuous monitoring applications.

\textbf{Robustness}: 100\% success rates under various contamination scenarios for reliable clinical decision support.

\textbf{Accuracy}: Low error rates for detailed analysis and validation studies.

\textbf{Scalability}: Efficient processing for large-scale clinical datasets.
