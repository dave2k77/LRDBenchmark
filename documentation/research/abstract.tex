Neurological time series analysis faces significant challenges in establishing reliable performance standards for long-range dependence estimation, particularly under realistic clinical confound conditions. This work presents a comprehensive benchmarking framework that addresses the reproducibility crisis in neurological time series analysis while establishing the foundation for physics-informed fractional operator learning approaches. We systematically evaluate 13 classical estimators across temporal, spectral, wavelet, and multifractal categories, alongside machine learning baselines, under eight realistic confound conditions including noise, outliers, trends, seasonality, missing data, smoothing, heteroscedasticity, and non-stationarity. Our framework demonstrates that machine learning approaches significantly outperform classical methods (R² > 0.96 vs negative scores) with 68\% better accuracy under confound conditions. We integrate experimental evidence from Physics-Informed Neural Operator (PINO) models achieving R² = 0.8802 with 205.5\% improvement over baseline methods, providing compelling validation for physics-informed approaches. Extended fractional calculus library benchmarks demonstrate exceptional performance improvements with 61.5x speedup for Marchaud derivatives and 35.4x speedup for Weyl derivatives, establishing computational feasibility for real-time applications. The framework achieves 100\% success rates across all tested estimators under realistic confound conditions, with CWT achieving 14.8\% average error and 0.009-second execution time. These results establish a new standard for neurological time series analysis while providing the complete computational foundation necessary for implementing physics-informed fractional operator learning in real-world clinical applications. The framework's robust performance under realistic conditions, combined with PINO validation and computational acceleration, positions it as a critical foundation for transformative neurological monitoring technologies.
