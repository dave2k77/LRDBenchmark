\section{Methodology and Experimental Design}

\subsection{Benchmarking Protocols and Procedures}

Our benchmarking methodology follows established protocols in computational neuroscience and machine learning evaluation \citep{Harris2020, Virtanen2020}. The experimental design ensures comprehensive evaluation under controlled conditions while maintaining realistic characteristics of neurological time series \citep{McKinney2010}.

\subsection{Data Generation and Preprocessing}

Synthetic data generation utilizes well-established models that capture the essential characteristics of neurological time series:

\begin{itemize}
    \item \textbf{Fractional Brownian Motion (fBm):} Generated using the Davies-Harte method
    \item \textbf{Fractional Gaussian Noise (fGn):} Derived from fBm processes
    \item \textbf{ARFIMA Models:} Autoregressive fractionally integrated moving average processes
    \item \textbf{Multifractal Random Walk (MRW):} Generated using wavelet-based synthesis
\end{itemize}

\subsection{Confound Application and Testing}

The confound testing protocol applies realistic clinical conditions systematically:

\begin{itemize}
    \item \textbf{Noise Addition:} Additive white Gaussian noise with varying SNR levels
    \item \textbf{Outlier Contamination:} Random replacement with extreme values
    \item \textbf{Trend Addition:} Linear and quadratic trends with varying slopes
    \item \textbf{Seasonality:} Periodic components with different frequencies
    \item \textbf{Missing Data:} Random data removal with interpolation
    \item \textbf{Smoothing:} Moving average filtering with varying windows
    \item \textbf{Heteroscedasticity:} Time-varying noise levels
    \item \textbf{Non-stationarity:} Piecewise stationary segments
\end{itemize}

\subsection{Machine Learning Baseline Implementation}

Machine learning estimators were implemented using scikit-learn and PyTorch frameworks \citep{Paszke2019, Abadi2016}:

\begin{itemize}
    \item \textbf{Neural Network (MLP):} Multi-layer perceptron with optimized architecture
    \item \textbf{Random Forest:} Ensemble method with 100 decision trees
    \item \textbf{Support Vector Regression:} SVR with RBF kernel
    \item \textbf{Gradient Boosting:} XGBoost implementation with early stopping
    \item \textbf{LSTM/GRU:} Recurrent neural networks for sequence modeling
\end{itemize}

\subsection{Performance Evaluation Metrics}

Comprehensive performance evaluation employs multiple metrics:

\begin{itemize}
    \item \textbf{Accuracy Metrics:} R² score, MAE, RMSE, MSE
    \item \textbf{Precision Metrics:} Standard deviation, confidence intervals
    \item \textbf{Efficiency Metrics:} Execution time, memory usage, speedup
    \item \textbf{Robustness Metrics:} Success rate, parameter sensitivity
\end{itemize}

% PLACEHOLDER: Insert Figure - Experimental Design
\begin{figure}[h]
\centering
\caption{Experimental Design and Methodology Flowchart}
\label{fig:experimental_design}
% INSERT FIGURE HERE
\end{figure}

\subsection{Statistical Analysis and Validation}

Statistical validation procedures ensure robust results:

\begin{itemize}
    \item \textbf{Monte Carlo Simulations:} 1000 realizations per condition
    \item \textbf{Cross-Validation:} 5-fold cross-validation for ML methods
    \item \textbf{Bootstrap Analysis:} 95\% confidence intervals
    \item \textbf{Hypothesis Testing:} t-tests, ANOVA, effect size analysis
\end{itemize}

% PLACEHOLDER: Insert Table - Experimental Parameters
\begin{table}[h]
\centering
\caption{Experimental Parameters and Specifications}
\label{tab:experimental_parameters}
% INSERT TABLE CONTENT HERE
\end{table}
